ラスコーリニコフは、もともと人ごみが苦手で、先に述べたように、とくに近ごろはどんな人づきあいも避けてきた。
ところがいま、彼はなぜかにわかに人々に惹きつけられた。
彼のなかで何か新しいものが立ちおこり、それと同時に、人恋しさにも似た飢えを感じたのだ。
彼は、まるひと月におよぶ凝縮された悩みと、陰うつな興奮のせいで疲れはてていたので、たとえいっときでも、たとえどんな場所でもいい、いつもとは別の世界で息がしたかった。
だから彼は、あたりの薄汚さにもめげず、いまは満足すら覚えながらこの居酒屋に腰をすえたのだった。

店の主人は別室にいたが、どこからか階段を降りてきては、しょっちゅう店内に顔を出していた。
そのたびにまず現れるのが、大きな赤い折り返しのある、靴クリームを塗った粋なブーツだった。
主人は、半外套をうえから羽織り、ノーネクタイで、黒い繻子のひどく脂ぎったチョッキを着こんでいたが、顔全体がまるで、オイルを塗りたくった鉄の錠前みたいな感じだった。
カウンターの向こうには、十四歳ぐらいのボーイと、それよりさらに年下のボーイがもう一人いて、注文があるたびに品物を出していた。
小ぶりのキュウリ、黒パン、魚の薄い切り身が並べてあり、ひどくいやな臭いを放っていた。
腰を下ろしているのも耐えがたいほどむし暑いうえ、すべてに酒の臭いがしみこんでいるので、空気を吸っているだけで、ものの五分もたてば酔いがまわってきそうだった。

まるで一面識もない相手ながら、ひと目みるなり、ひとことも言葉をかわさないうちから、なぜか急に興味をそそられるような人との出会いが、ときとしてある。
ラスコーリニコフから少し離れてすわっている、退職役人風の客から受けた印象というのが、まさにそんなふうなものだった。
青年は、あとになんどかこのときの最初の印象を思い起こしては、あれこそ虫の知らせだったと思った。
彼は、ちらりちらりと絶えずその役人に目をやったが、それは、その役人がしきりにこちらを見つめ、声をかけたくてうずうずしているらしいせいもあった。
役人は、店の主人もふくめ、居酒屋にいる残りの連中になぜだか慣れっこになっていて、いかにも退屈そうに、と同時に、身分も教養も低すぎてとても話し相手にならないとでも言わんばかりの、何やら偉ぶった、見くだすような目で彼らを眺めていた。

すでに五十を越えようかという、中背ながらがっしりした体つきの男で、白髪まじりの頭にはおおきなはげがあり、アルコール漬けのせいでむくみのきた、黄ばんだ、いやむしろ青みがかった顔をし、腫れぼったい瞼の奥からは、何か裂け目のように小さい、それでいて燃えたつような、赤みをおびた目が輝いていた。
しかし、男にはどことなく、じつに奇妙なところがあった。
そのまなざしには、何か感激性といったものさえ輝いているのだが――おそらく分別も知恵もあるにはあるのだろう――と同時に、そこに何やら狂気のような感じが煌めいているのだ。
男は、ボタンもとれた、古い、すっかりぼろぼろになった黒の燕尾服を着ていた。
ひとつだけかろうじてついているボタンを、どうやら礼儀を失するまいと願うらしく、律儀にかけていた。
南京木綿でできたチョッキの下からは、しわくちゃになった、しみだらけの汚らしい胸当てがはみ出ていた。
顔は役人風に剃ってあったが、それもだいぶ経っているようで、青みがかった、ごわごわしたひげがびっしり伸びかけていた。
それに彼のしぐさには、事実、いかにも役人らしくものものしい趣があった。
そのくせそわそわと落ち着きがなく、髪の毛をかきむしったり、こぼれた酒でべとついたテーブルに穴の開いた両肘をついて、いかにも所在なげに両手で頭を抱えこんだりしていた。
やがて彼は、ラスコーリニコフのほうをまともに見すえ、大きなしっかりした声で話しかけてきた。

「失礼ですが、そこのお方、ひとつまじめなお話をさせてはもらえませんかね?
といいますのも、見栄えこそぱっとしておられませんが、そこは年の功、あなたには学もあり、酒もあまり飲み慣れていないお方ってことがすぐにわかるんでして。
わたし自身、誠意とひとつにむすびあった教養といったものを日ごろから重んじておりまして、そればかりか、九等官の末席を汚しておる身でしてね。
マルメラードフ、そういう苗字で、九等官をしております。
で、失礼ですが、お勤めでいらして?」\\
「い、いえ、学生です……」
青年は、相手のもってまわったような口ぶりや、あまりにもストレートに話しかけられたことにもいくぶん面くらって、答えを返した。
ついさっき、どんな相手でもいい、人と話をしてみたいと一瞬願ったのが嘘のように、いざ、こうして声をかけられてみると、自分の一身にふれる、あるいは少しでも触れようとする他人への、いつもながらの不快で、いらだたしいほどの嫌悪感にかられるのだった。

「てことは、学生さん、それとも元学生さんってわけですな!」
役人は叫んだ。
「思ったとおり!
年の功ってやつですよ、おにいさん、これが年の功ってもんなんでして!」
そう言うと彼は、いかにも得意げに指を一本、額に押し当てた。
「学生さんだったわけね、でなくとも、学問の道を歩いてこられた!
では、失礼ながら一つ……」
そう言って彼は立ちあがり、そのはずみでぐらりとよろけたが、ウォッカのびんとコップをつかんで、青年の斜向かいの席に腰を下ろした。
酔ってはいたが、話しぶりは勢いがあって、雄弁だった。
ところどころいくらか言葉につまって、「えー」をくりかえした。
まるひと月、だれとも話をしてこなかったかのように、何かしら貪るような調子でラスコーリニコフにからんできた。

「で、よいですかな」
彼はほとんどもったいぶった調子で切り出した。
「貧乏は悪徳ならず、こいつは真理ですな。
わかっとりますとも。
酒が徳ならずってことぐらい、まして、ね。
でも、これが極貧となったらです、極貧となったら、こいつはもう悪徳なんでございますな。
たんに貧乏なだけなら、生まれながらの上品な気持ちを保っておられますが、これが極貧となったら、だれだってそうはいきませんよ。
つまり、極貧ってことになったら、こいつはもう棒っきれで追っぱらわれるどころじゃない、もっと恥ずかしい思いをさせてやろうってんで、箒で掃かれ、人間のお仲間からぽい捨てされちまう。
しかも、それが当然なんですよ。
なぜって、極貧ってことになれば、自分でまっさきに自分を辱めにかかりますからなあ。
で、行きつく先は酒場通いってことになるわけです!
で、よいですかね、今からひと月前です、うちの家内がレベジャートニコフ氏にこっぴどく殴られましてね。
ですが、うちの家内はわたしなんぞとはちょいと出来が違う!
おわかりですかな?
それともうひとつ、ほんの好奇心ってことで質問させていただきますがね、あなた、ネヴァ川の干草船ってところで、ひと晩明かしたことがございますか?」\\
「いや、ありませんよ」とラスコーリニコフは答えた。
「それって、いったいなんの話なんです?」
「いえね、わたしはそこから通っとるんですよ、もう五泊目になりますか……」

彼はコップに酒を注ぎ、一気にそれをあおると、考えこんだ。
たしかに、彼の上着といわず髪の毛にまで、干草がちらほら貼りついているのが見えた。
彼がもう五日間、着替えをせず、ろくに顔も洗っていないことは歴然としていた。
とくに手がひどい汚れようで、脂ぎり、赤みをおび、爪は真っ黒だった。

どうやら彼の話は、酒場にいる客たちのけだるい興味を呼びさましたらしかった。
カウンターの向こうにいるボーイたちは、くすくすしのび笑いを漏らし始めた。
店の主人は、この「おもろい男」の話を聞こうとわざわざ上の階から降りてきたらしく、いかにもかったるそうに、そのくせもったいぶってあくびなど漏らしながら、少し離れたところに腰を掛けた。
マルメラードフは、あきらかにここの常連らしかった。
かれのものものしい話しぶりは、見知らぬいろんな客を相手に、頻繁にくだを巻いているうちに身についたらしかった。
こういう習慣は、ある種の酒飲みにあっては、根っからの欲求と化しているものなのだ。
家できびしく扱われたり、こき使われている連中の場合がとくにそうである。
だからこそ、彼らは、せめて酔っぱらい仲間には何とか自分の言い分を聞いてもらい、できれば尊敬までも勝ち取ろうとつねにやっきになるのだ。

「おい、そこのお調子もん」
店の主人が大声で叫んだ。
「役人のくせして、なんだって働かない、なんだって勤めに出ない?」\\
「わたしがどうして働かないかって、だんな?」
マルメラードフは、まるでラスコーリニコフから水を向けられたとでもいうように、もっぱらそちらに顔を向けながら、話を引きとった。
「どうして勤めに出ないか?
じゃ、こんなむだな暮らししてて、わたしの心が痛まないとお思いですか?
ひと月前、レベジャートニコフ氏がうちの家内を殴りつけたときも、わたしは酔っぱらって寝ていましたがね、それでわたしが苦しまなかったとでも?
失礼ながら、学生さん……、たとえば……そう……絶望的借金ってのを、しようとなすったことがおありですか?」\\
「ありますよ……
でも、どう絶望的なんです?」\\
「つまり、まるきり絶望的なんでございますよ。
借金を申し込んだからってどうにもならないのが初めからわかっている。
たとえば、そう、この男、このたいそう高潔ですこぶる有益な市民がです、まかりまちがっても金など貸してくれないとわかっている。
だって、そうでしょうが。
こっちこそ聞きたいくらいです。
どうして貸してくれるのか、ね?
なにしろ相手は、こっちが金を返さないことくらい百も承知なんですよ。
同情から?
でも、新思想を追っかけているレベジャートニコフ氏が、ついこの間も説明してくれましたっけ。
現代じゃ、同情なんてもんは学問上も禁じられてる、経済学とかいうのが発展しているイギリスじゃ、現にそういうふうになってる、って。
どうして金を貸してくれるのか、こっちこそ聞きたい。
ところがです、貸してくれるはずがないと初めからわかってるくせに、それでも、のこのこ出かけていく、で……」
\\「じゃ、どうして出かけていくんです?」
ラスコーリニコフは口をはさんだ。
\\「でも、もし行く相手がなかったら、これ以上、行き先がなかったら、どうなるか!
人間だれしも、どこかに行き先がなくっちゃ、どうしようもない。
何せ、どこでもいい、どこかに行かなくちゃならないときってのがあるもんなんですから!
うちの一人娘がはじめて黄の鑑札のお世話で仕事に出ていったときは、さすがのわたしも出かけていきましたよ……
(何せうちの娘はこの鑑札で暮らしてるんでございまして……)」
と彼は、いくらか不安げな面持ちで青年を見やりながら言い添えた。