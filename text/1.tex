七月の初め、異常に暑いさかりの夕方近く、ひとりの青年が、S横町にまた借りしている小さな部屋から通りに出ると、何か心に決めかねているという様子で、ゆっくりとK橋のほうに歩きだした。

階段口で彼は、下宿のおかみと無事顔を合わさずにすんだ。
彼が借りている小部屋は、五階建ての高い建物の屋根の真下にあって、部屋というよりもどこか戸棚を思わせるところがあった。
食事と女中つきで彼に部屋を貸している当のおかみは、一階下の独立した部屋に住んでいたので、外出のたびに彼は、階段に向かってほとんどいつも開け放たれている台所の脇を、いやでも通らなくてはならなかった。
そしてそこを通るごとに、何か病的ともいえる気後れにかられ、そのことを自分でも恥ずかしく感じて、そのためにまた顔をしかめるのだった。
下宿代がたまりにたまっていたので、おかみと顔を合わせるのが怖かったのである。

かつて彼は、こんな風にも臆病でいじけた青年ではなかった。
いやむしろ、それと正反対なぐらいだった。
ところがいつの時点からか、心気症にも似た、いらだちやすい、張り詰めた状態に陥っていた。
あまりに深く自分の殻に閉じこもり、世間の人たちからも孤立してしまったため、下宿のおかみどころか、相手がだれであれ、人と顔を合わせるのが怖くなった。
貧乏にも押しひしがれていた。
ところが、近ごろは、そんなせっぱつまった暮らしにも苦しさを感じなくなっていた。
毎日の差し迫った仕事を丸ごと放り出し、それにとりかかる気にもなれなかった。
だから、じつのところ、自分にどんな魂胆を抱いていようと、下宿のおかみ風情などへとも思っていなかったのだ。
ただし、階段口で呼び止められ、自分にはなんの関わりもないばかげた世間話やら、いつものしつこい下宿代の催促やら、脅しやら、泣き言やらを聞かされると、こちらもうまく返答をかわしたり、わびのひとつも入れたり、嘘をついたりしなければならず——いや、そんなことなら、いっそ子猫みたいに、忍び足で階段をすり抜け、だれにも見られないようにこっそり逃げだすほうがはるかにましだった。

しかし今日ばかりは、表に出るなり、債権者のおかみと顔を合わせるのをここまで怖れていたかと、われながらあきれかえった。

《あんな大それたことを決行しようとしているのに、こんな愚にもつかぬことにびくついたりして!》
奇妙な含み笑いを浮かべながら、彼は思った。
《なるほど……
そういうことか……
人間ってのはすべてを手中に収めながら、それをみすみす逃してしまう、それももっぱら臆病のせいで……
こいつはもう公理といってもいいぞ……
おもしろいのは、人間がいの一番に怖れるものって何かってことだ。
新しい一歩、自分の新しい言葉、人間は何よりもそれを恐れているんだ……
それにしても、おしゃべりがすぎるな。
何もせずにいるのは、このおしゃべりのせいだ。
いや、逆にこういうことかもしれん。
おしゃべりがすぎるのは、なにもしていないからだ。
一日中、下宿に寝転がって……
そう、ゴロフ王のことなんか考えながら、こうしてしゃべることを覚えたのはついこのひと月じゃないか。
ところで、どうしておれはいま歩いている?
ほんとうにおれにあれができるのか?
いったいあれは本気なのか?
なあに、本気なわけがあるもんか。
そうさ、空想で、自分で自分を慰めているだけさ、おもちゃだな!
そうさ、どうやら、おもちゃってところが正解らしいぞ!》

通りはひどい暑さで、しかも息づまるような熱気と雑踏、あたり一面の漆喰、建築の足場、れんが、土ぼこり、そして、別荘を借りる余裕のないペテルブルグっ子なら誰もが知る、あの、夏特有の悪臭——これらすべてがたちまち、そうでなくても調子の狂った青年の神経を、不快にかき乱した。
ペテルブルグのこの界隈に特に多い居酒屋から流れてくるたまらない悪臭と、平日にもかかわらずひっきりなしに出くわす酔っぱらいたちが、胸くそ悪くなるような陰惨な町の光景に、最後の色どりを添えていた。
途方もなく深い嫌悪感が、一瞬、青年の端正な顔立ちをかすめた。
ついでに言っておくと、青年は、なかなかの美男子だった。
黒く美しい目、栗色の髪、背丈は平均よりも少し高く、やせぎすですらりとしていた。
けれども彼は、やがて深いもの思いに沈んでしまった。
より正確には、何やら夢うつつの状態におちいったらしく、周囲のことなど何ひとつ気にとめず、というより気にとめたくないと念じながら歩きだした。
ときおり彼は、つい今自分でも認めた独白癖から、ぶつぶつひとりごとを重ねることもあった。
そしてこの瞬間、自分の考えがときおり混乱してしまうことや、体がかなり衰弱していることも自覚していた。
これでもう二日、ほとんど何も口にしていなかったのだ。

彼の身なりはあまりにひどいもので、ほかの、たとえそういうのに慣れっこになった人間でさえ、こんなぼろ服をまとって真っ昼間に外出するなど、とても恥ずかしくてできなかっただろう。
もっともこの界隈は、身なりで人をおどろかすといったことがまずはむずかしい、そんなふうな地区だった。
センナヤ広場の近くにあって、いろいろと怪しげな遊び場も多いし、それに、ペテルブルグの中心街になるこのあたりの表通りや裏通りには、とりわけ工員や職人が密集し、あれこれ風変わりな連中が町全体の風景を色どっていたので、そんな他人と顔を合わせるたびにいちいちおどろいたりすれば、それこそおかしなことになりかねなかった。

それに、青年の心は、敵意にも似た軽蔑の念が溜まりに溜まり、もともとがごくデリケートで、時には初々しいほど敏感ながら、今はこうしてぼろぼろの服で外出することすら、少しも恥ずかしいとは感じなかった。
もっとも、日ごろから会いたくないと思っている知人やかつての学友に出くわすともなれば、それはまた別の話だった……
と、そのとき、こんな昼どきにどこに何しにお出ましになるのか、ばかでかい駄馬をつないだ大きな荷馬車に乗ったひとりの酔っぱらいが、通りしな、いきなり彼をどやしつけた。
「おい、そこのドイツ・シャッポ!」
そして、手で彼をさし示しながら、声をかぎりにどなりはじめた。
青年はぎくりとして立ち止まり、帽子を慌ててひっつかんだ。
帽子は、丈の高いツィンメルマン製の丸帽だったが、もうすっかり古びて色あせ、
穴としみだらけでつばもとれ、おまけに角の部分がおそろしくぶざまにひしゃげて、横に飛びだしていた。
それでも、彼をとらえたのは、恥ずかしさというより、むしろ驚きに似たまるきりべつの感情だった。

《やっぱりそうだろ!》
どぎまぎしながら、彼はつぶやいた。
《思っていたとおりだ!
こいつがいちばんあぶないんだよ!
それ見ろ、こういうくだらんこと、こういう些細なことから、計画がすべておじゃんになるんだ!
それにしても、目立ちすぎる帽子だ……
こっけいだから、よけい人目についてしまう……
おれのこのぼろ服に合わせるのは、ぜったいに学生帽でなきゃだめだ、たとえしなびた煎餅みたいなやつでもいい。
こんな化けものじみた帽子じゃなくて。
こんなもの、今どきだれもかぶってないから、一キロ先からだってすぐ目につくし、覚えられちまう……
要するに、後々まで覚えられてしまったら、それだけで立派な証拠にもなるってことだ。
なるべく人目を引かないようにしなきゃ……
些細なこと、些細なことこそ大事なんだ!……
この些細なことってのが、いつもすべてをぶちこわしにしちまうんだから……》

距離はいくらもなかった。
家の門から何歩かということも知っていた。
ちょうど七百三十歩だった。
この空想にすっかりふけり出したところ、彼はいちど数えてみたことがあったのだ。
そのころはまだ、自分でもその空想を信じられず、醜悪ながらもその魅惑的な大胆さに、いらだちを募らせていただけだった。
ところがひと月たった今、彼はもうべつのちがった目で見はじめており、例の独白癖で自分の無力さや優柔不断をからかいつづけながらも、あの《醜悪な》空想を、なにかいやおうない、既定の事業と考えることに慣れてしまっていた。
しかしそのじつ、自分でもまだ信じることができないでいたのだ。
今も彼は自分の事業のリハーサルのために歩いており、一歩ごとにいよいよ興奮が高まっていった。

心臓をどきどきさせ、体を小刻みにふるわせながら、一方の壁面が運河に面し、もう一方が***通りに面した、とてつもなく大きな建物へと向かっていった。
この建物は、全体が細かい部屋に区切られていて、仕立て屋から金物工といったあらゆる職種の職人や、料理女、いろんなドイツ人、春をひさぐ女たち、小役人その他が入居していた。
建物の二つの門と、二つの中庭は、ひっきりなしに人の出入りがあった。
そこには、三、四人の庭番が勤めていた。
青年は、そのうちのだれとも顔を合わさずにすんだことにいたく満足しながら、目立たないよう門からすっと右に折れて階段に向かった。
階段は、暗くて狭いいわゆる《裏階段》だったが、こうしたこともすべて知りつくしていたし、きちんと頭にも入っており、むしろそうした状況がすべて気に入っていた。
これぐらい暗ければ、好奇の目も怖れるには足りなかった。
《いまからこんなにびくついているようでは、いざ、あれを決行するとなった段には、いったいどうなることやら?……》
四階につづく階段を昇りながら、彼は思わず考え込んだ。

と、そこで、部屋から家具を運び出していた兵隊あがりの運送屋に、行く手をふさがれた。
その部屋にドイツ人の役人一家が住んでいることは、前々から知っていた。
《てことは、ここのドイツ人は引っ越すわけだ。
つまり四階のこの会談とこの踊り場は、これからしばらくあのばあさんの専用ってことになる。
こいつは悪くないぞ……
万が一ってこともあるし……》
彼はまたこう考え、老女の部屋の呼び鈴のひもを引いた。
呼び鈴は、銅ではなくブリキでできているかのように、ガランと力なく鳴った。
こういう建物のこういう部屋には、たいてい、こんなふうな呼び鈴がついている。
彼はその呼び鈴の音を忘れてしまっていたが、今この独特の音がふと何かを思い出させ、まざまざとそれを目に浮かべたかのようだった……。
そこで、思わずぎくりとなった。
前とはちがって、このときばかりはもうあまりにも神経が衰弱してしまっていたのだ。

しばらくして、ドアがほんのわずかにあいた。
部屋の女主人は、いかにもうさん臭そうに、すき間から来客をじろりと見まわした。
こちらからは、暗がりに光る二つの目だけが見えた。
しかし、踊り場にいくつか人影があるのを見て心強く思ったか、やがてすっかりドアを開けはなった。
青年は敷居をまたぎ、衝立で仕切られた暗い玄関口に入った。
衝立の後ろには、ちっぽけな台所があった。
老婆は、だまりこくったまま青年の前に突っ立ち、不審そうに相手を眺めていた。
やせた、小柄な老女だった。
年のころ六十前後、悪意のこもるするどい目つきをし、鼻はちいさくとがり、頭には何もかぶっていなかった。
白髪のまじる薄色の髪には、油がたっぷり塗ってあった。
にわとりの脚のように細長い首には、フランネルのぼろ布のようなものが巻いてあって、この暑さだというのに、肩に擦り切れて黄色く色あせた毛皮の胴着を羽織っていた。
老女はひっきりなしに咳き込み、のどを鳴らしていた。
おそらく自分を見つめる青年のまなざしに、なにか一種特別のものがあったのだろう、老女の目にも、さっきの不審のいろがふいにちらりと浮かび上がった。

「ラスコーリニコフですよ、学生の。
一カ月前にもうかがったんですがね」
もっと愛想よくしなければと思いかえし、青年は軽く会釈をするとあわててつぶやいた。\\
「覚えてますよ、以前にいらしたのはよく覚えてます」
不審そうな目をあいかわらず相手の顔から離さず、老女ははっきりと言いはなった。\\
「それでですね……また用件があって来たんですよ……」
老女の疑り深さに驚きながら、ラスコーリニコフはいくぶんどぎまぎしてつづけた。

《だが、もしかするとこの女はいつもこうで、あのときはそれに気づかなかっただけかもしれない》
不快な感情にかられながら、彼はそう思った。

老女は何か思案するふうにしばらく黙っていたが、やがて脇に身を引くと、部屋のドアを指さし、客を先に通しながら言った。\\
「お入りなさいな、おにいさん」

青年が通された小さな部屋は、黄ばんだ壁紙が張られ、窓にはゼラニウムの鉢植えがおいてあり、モスリンのカーテンが掛かっていたが、ちょうどこのとき、夕日に明るく照らしだされていた。
《てことは、きっとあの時も、太陽がこんなふうに照らしだすんだな!……》
ラスコーリニコフの脳裏にはからずもこんな考えが浮かび、できる限り家具の配置を覚えておこうと、室内にあるすべてのものに素早く視線を走らせた。
だが、室内にこれといって目ぼしいものはなかった。
家具はどれもこれも古い黄木製のもので、大きくそり返った背もたれのあるソファ、その前に置いてある楕円形のテーブル、窓と窓の間には鏡のついた化粧台、壁ぎわには数脚の椅子、それに、小鳥をてのひらに乗せたドイツ娘を描く黄色い額入りの安物の絵が二、三枚——それだけだった。

部屋の隅にある小さな聖像の前に灯明が灯っていた。
すべてがいたって清潔で、家具も床もつや出しがかけられていた。
何もかもがぴかぴかに光っていた。
《リザヴェータがやってるんだ》
青年はふと思った。
部屋じゅうどこを見まわしても、ちりひとつ落ちていなかった。
《ごうつくばりの年寄り後家さんの家にかぎって、だいたいがこんなふうにきれいなんだ》
ラスコーリニコフは心のうちでそうつぶやきながら、奥の、ごく小さな部屋に通じるドアの前にかかった更紗のカーテンを横目で見やった。
そこには老女のベッドと箪笥が置いてあったが、まだいちどもそちらをのぞいたことはなかった。
老女の住まいは、この二部屋ですべてだった。\\
「何のご用です?」
部屋に入ると老女は、相手の顔をじかに見ようと、さっきと同じように彼の真ん前に突っ立ったまま、きびしい口調でたずねた。\\
「質草を持ってきたんですよ、ほら、これです!」
そう言うと彼は、平たくて古い銀時計をポケットから取りだした。
時計の裏蓋には地球儀が描いてあった。
鎖は鋼鉄でできていた。\\
「そう、前回質入れなさった品ですがね、あれももう期限ですよ。一昨日でひと月立ちましたがね」\\
「利子をもうひと月分、お支払いします、もう少し辛抱願います」\\
「言っときますがね、お客さん、辛抱するか、すぐに流しちまうか、そりゃあたしの勝手ですがね」\\
「この時計はいい値がつくでしょう、アリョーナさん?」\\
「ろくでもない品ばかり持ってきて、お客さん、だめですよ、ろくな値打ちもありゃしません。
前回の指輪には二ルーブルおつけしときましたがね、あれだって、宝石屋に行きゃ、新品が一ルーブル五十コペイカそこそこで買えるんだから」\\
「四ルーブルぐらいつけてもらえませんか、親父の形見ですから、かならず請けだします。
もうすぐお金が入ることになっていますし」\\
「利子天引きで、一ルーブル五十コペイカでどうかね?」\\
「一ルーブル五十コペイカだって!」
青年は思わず声を上げた。\\
「いやならいやで結構ですがね」
そういって老女は、時計を相手に突き返した。
青年はそれを手に取り、腹立たしさのあまり直ちにその場を立ち去ろうとしたが、すぐ思いとどまった。
ほかに行くあてもなかったし、それに自分にはまた別の用件もあったことを思い出したのだ。\\
「それで結構です!」
乱暴にそう言いはなった。

老女はポケットに手を入れて鍵束を探ると、カーテンの奥の部屋に入っていった。
一人部屋の真ん中に残された青年は、好奇心にかられて聞き耳を立て、あれこれ考えをめぐらせた。
箪笥の鍵を開ける音が聞こえた。
《きっといちばん上の引き出しだな》と考えた。
《鍵束は、つまり、右のポケットに入れてるってわけだ……
鉄の輪でひと束にしてな……
そのうち、鍵歯のついたいちばん大きな、ほかより三倍くらい大きいのがひとつあったが、あれはむろん、箪笥のじゃない……
ってことは、ほかにもまだ手箱かトランクがあるってことだ……
こいつはおもしろいぞ。
トランクにはたいてい、ああいった鍵が……
ああ、それにしても、すべてがなんてあさましい……》

老女が戻ってきた。\\
「ほら、お客さん、一ルーブルにつき月十コペイカを利子にして、一ルーブル半だから、先払いで十五コペイカを天引きさせてもらいますよ。
それに前回の二ルーブルは今のところ二十コペイカ。
つまり合計で三十五コペイカ。
てわけで、お客さんの時計の受け取り分は、しめて一ルーブル十五コペイカ。
さあ、受けとんなさいな!」\\
「えっ、合計で一ルーブル十五コペイカ!」\\
「ええ、その通りですよ」

言い争うのをやめ、青年はおとなしく金を受け取った。
老女を見つめたまま、彼は帰りを急ごうとしなかった。
なんかまだ言いたりないことが、したりないことがあるような気がしたが、自分でもそれがいったい何なのかわからない様子だった…。\\
「アリョーナさん、もしかしたら、数日中にまた質草を持ってきます……
銀製の……なかなかいいタバコ入れでしてね……
友人から返してもらえたらすぐ……」
彼はどぎまぎし、そのままだまり込んだ。\\
「それはまた、そのとき話しましょうよ、お客さん」\\
「じゃあ、これで……
そうそう、おばあさんはいつもお宅に、おひとりでおられるんですね、妹さんはいらっしゃらなくて?」
玄関に向かいながら、できるだけくだけた調子でたずねた。\\
「お客さん、妹にどんな用がおありで?」\\
「とくに何もありませんがね。
ただ聞いてみただけですよ。
それをすぐに……
じゃあ、また、アリョーナさん!」

ラスコーリニコフは、完全に狼狽しきって部屋を出た。
狼狽は、いよいよはげしさを増していった。
階段を下りる途中、何かにはっと怯えたかのように、何度も足を止めたほどだった。
そして通りに出たところで、とうとう声をはりあげた。\\
「ああ!
何もかもむかつく!
ほんとうに、ほんとうに、おれは……
いや、あんなものはナンセンスだ、たわごとだ!」
彼は決然と言い放った。
「ほんとうに、どうしてあんな怖ろしい考えが頭に浮かんだんだ?
それにしても、おれの心は、なんてきたないことを受け入れることができるんだ!
なにより、きたないこと、汚らわしいこと、下劣なこと、そう、下劣なことだ!……
おれは、まるひと月……」

しかし、言葉によっても叫びによっても、心の高ぶりを言いあらわすことはできなかった。
老女の家に向かう途中から、自分を押しつぶし、苦しめはじめた果てしもない嫌悪感が、今や怖ろしいほどの大きさに達し、あまりにもくっきりと正体を明らかにしたため、自分でももう、その苦しさからどう逃げ隠れしてよいものか、わからなくなっていたのだ。
まるで酔っぱらったように、行きかう人々にも気がつかず、人にぶつかりながら歩道を歩いていた。
そして、次の通りに入ってようやく正気にもどった。

あたりを見まわし、自分が居酒屋のすぐそばに立っていることに気づいた。
居酒屋の入り口は、歩道から階段を下りた地下にあった。
そのときちょうど、ドアからふたりの酔っぱらいが出てきて、たがいにもたれあい、罵りあいながら通りに上がってきた。
ラスコーリニコフは、長く考えることなく、すぐに階段を降りて行った。
これまで、こうしたたぐいの居酒屋にいちども足を踏み入れたことがなかったが、いまはくるくる眩暈がしていたし、おまけに焼けるような喉の渇きに苦しめられていた。
冷えたビールをぐいとひと飲みしたかったし、しかも急に襲ってきた体力の低下を彼は空腹のせいとも考えていたのだった。

暗く、汚らしい居酒屋の隅に腰を下ろし、べとべとするテーブルに向かった彼は、ビールを注文し、最初の一杯をむさぼるように飲み干した。
するとたちまち気分が楽になり、頭の中もはっきりしてきた。
《何もかもくだらない》
彼は、心に光を感じながらそう口にした。
《どぎまぎする理由がどこにあったんだ!
体調が悪かっただけのことだ!
ビールをグラスで一杯と、乾パンのひとかけらで――ほらこのとおり、頭はたちまちしっかりし、考えもはっきりする、計画もしゃきっとしてくる!
ぺっ、何もかもくだらんことばかり!……》
だが、こうして蔑むように唾を吐いてはみたが、何か怖ろしい重荷から急に解きはなたれたかのようにすっかり明るい顔になり、まわりに居合わせる人々に愛想よく目を走らせはじめた。
しかしこの瞬間さえ、彼はこの、何ごともよくとろうとする感覚そのものもまた病的なのだと、ぼんやり予感していた。

このとき、居酒屋にはごくわずかな客がいるだけだった。
階段で出くわしたふたりの酔っぱらいのあとから、女連れの五人ばかりの男の一団が、アコーディオンを抱えてどやどやと出て行った。
彼らがいなくなると、店内は急に静かになり、広々とした感じになった。
残っていたのは、ビールと向かい合って腰をかけている、ちょっと見た目には町人風の、軽く酔いがまわった感じの男と、その連れでシベリア帽をかぶり、白髪まじりのあごひげを生やしている、でっぷり太った大柄な男だった。
こちらはひどく酔っていて、椅子の上でうとうとしていたが、ときおり寝ぼけたようにとつぜん両手を大きく広げて指を鳴らし、椅子から腰を上げずに上半身でひょいと跳びあがるまねをしたりした。
そして、必死になって歌詞を思い出そうとしながら、何かばかくさい歌を口ずさむのだった。

  まる一年、にょーぼーをかわいがった
  まある一年、にょーぼーを、かーわいがった……
  かとおもうと、いきなり目をさまして、またもや――

  ポジヤチェスカヤ通りを歩きだしたら、
  昔のにょーぼー、見っけた……

だが、その男のおめでたい気分をともにわかってやろうという相手は、だれひとり現れなかった。
むっつり屋の彼の連れは、相手の突発的なしぐさを、むしろいまいましげに、うさん臭そうな目で眺めやっていた。
居酒屋にはほかに、見たところ役人あがりといった風采の男がもうひとりいた。
彼はウォッカの子びんを前に、ひとりぽつねんと腰かけ、ときたま一口また一口とやっては、ぐるりとあたりを見まわしていた。
彼もまた、いくぶん興奮ぎみの様子だった。 